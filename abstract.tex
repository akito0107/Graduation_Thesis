\begin{center}
\textbf{\Large 卒業論文要旨 2014年度(平成26年度)}

\vspace{6.18mm}

\textbf{\Large T-Ring:センサデータの時間的特殊性に着目した多次元データ分散管理システム}
\end{center}

\vspace{10mm}
近年,センサネットワークは環境モニタリングや,スマートグリッドなど,我々の生活を豊かにするインフラとして整備されつつある.また,センサノードの小型化やセンサを搭載したスマートフォンの普及により,センサ自体やセンサネットワークが,家庭,個人の規模で利用されつつあり,センサネットワークが我々を支える生活基盤と成り得る将来は遠くはない.

本研究は,センサネットワークがより一般的な生活基盤となり,他の組織や家庭などとデータを公開,共有し,センサの具体的な所在を気にすることなく,自由にデータを利用できる世の中を想定する.この環境下で,公開,共有されたセンサデータを公衆センサデータとする.

この公衆センサデータの管理を行うには,センサデータを緯度,経度,センサタイプを含んだ多次元のデータとして扱わなければならないが,従来の多次元データ管理の研究において,センサデータを対象とした研究は数が多くない.また,センサデータを対象にしていても,センサデータの時間的特殊性という性質に着目している研究は存在しない.

本研究は,このセンサデータの時間的特殊性に着目した多次元データ管理システムであるT-Ringを提案する.T-Ringは時間を多次元データにおける1つの属性とせず,保存先変更の基準として扱うことで,データの保存,取得時の計算量を減らし,処理時間を減少させる.さらに,本研究ではこのT-Ringシステムの性能を評価することで,有用性を証明する.

\vspace{10mm}
キーワード :\\
\hspace{3.5em}公衆センサデータ, 多次元データ管理,P2P,分散ネットワーク,構造化オーバーレイネットワーク

\begin{flushright}
\textbf{慶応義塾大学環境情報学部}\\
\textbf{小町芳樹}
\end{flushright}

\newpage

\begin{center}
\textbf{\Large Abstract of Bachelor's Thesis}
\textbf{\Large Academic Year 2014}
\vspace{6.18mm}

\textbf{\Large T-Ring:Sensor Data Oriented Decentralized Multidimensional Indexing Focusing on Particularity of Time}
\end{center}

\vspace{10mm}

In recent years, sensor networks are being developed as an infrastructure to enrich our lives such as environmental monitoring and smart grids. In addition, by the spread of sensor equipped smartphones and downsizing of sensor nodes, sensor networks (or the sensor itself) are being used in the scale of family or personal. In the future, sensor networks will become the foundation of our life supporting. 

This research assumes that sensor networks could become the foundation and that everyone could get data from anywhere without being concerned about which sensor network they belong to. In this environment, these published and shared sensor data are called "public sensor data" in this paper. 

To manage this public sensor data, sensor data should be operated as Multidimensional Indexing which has latitude, longitude, sensor type and so on. However, in general works of Multidimensional Indexing, these researches were hardly intended on sensor data. No reserach focused on particularity of sensor time.   


This research proposes T-Ring which is Multidimensional Indexing system for public sensor data. This is the first research focusing on particularity of sensor time. T-Ring doesn't use time attributtion in multidimensional data and  use a criterion of deciding the place of storage. T-Ring reduces amount of calculations on storing and retrieving of data and processing speed. This paper evaluates this system and proves feasibility.   

\vspace{10mm}
Keywords :\\
\hspace{3.5em}Public Sensor Data, Multidimensional Indexing, P2P, Decentralize Network, Structured Overlay Network
\begin{flushright}
\textbf{Yoshiki Komachi}\\
\vspace{5mm}
\textbf{Faculty of Environment and Information Studies}\\
\textbf{Keio University}
\end{flushright}
